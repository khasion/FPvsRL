\documentclass{article}
\usepackage{graphicx}

\title{Report on Game Simulation Results}
\author{}
\date{}

\begin{document}

\maketitle

\section*{Introduction}
This report summarizes the results of simulations for two strategic games: Rock-Paper-Scissors and Prisoner's Dilemma. The simulations compare the performance of two algorithms, Fictitious Play (FP) and Q-Learning (QL), over 1000 episodes each. Key metrics include the number of wins for each algorithm and the number of draws.

\section*{Results}

\subsection*{Rock-Paper-Scissors}
\begin{itemize}
    \item \textbf{Total Episodes:} 1000
    \item \textbf{Fictitious Play Wins:} 90
    \item \textbf{Q-Learning Wins:} 389
    \item \textbf{Draws:} 521
\end{itemize}

\begin{figure}[h!]
\centering
\includegraphics[width=0.8\textwidth]{rps_results.png}
\caption{Results for Rock-Paper-Scissors}
\end{figure}

\subsection*{Prisoner's Dilemma}
\begin{itemize}
    \item \textbf{Total Episodes:} 1000
    \item \textbf{Fictitious Play Wins:} 98
    \item \textbf{Q-Learning Wins:} 200
    \item \textbf{Draws:} 702
\end{itemize}

\begin{figure}[h!]
\centering
\includegraphics[width=0.8\textwidth]{prisoners_results.png}
\caption{Results for Prisoner's Dilemma}
\end{figure}

\section*{Discussion}
The results highlight key differences in the performance of the algorithms across the two games. In Rock-Paper-Scissors, Q-Learning significantly outperforms Fictitious Play, while in Prisoner's Dilemma, the difference is less pronounced. The number of draws is higher in the Prisoner's Dilemma, indicating potential equilibria between strategies.

\section*{Conclusion}
These simulations provide insights into the behavior of Fictitious Play and Q-Learning in strategic games. Future work could involve extending the analysis to more complex games and incorporating additional metrics such as convergence speed.

\end{document}